\documentclass[
    pdftex,             % PDFTex 
    a4paper,            % A4 Papier
    oneside,            % einseitig
    liststotoc,         % Verzeichnisse einbinden in toc
    halfparskip,        % Abstand zwischen Absätzen
    chapterprefix,      % Kapitel ausschreiben
    headsepline,        % Linie nach Kopfzeile
    %footsepline,       % Linie vor Fusszeile
    pointlessnumbers,   % Nummern ohne abschließenden Punkt
    12pt                % Schriftgröße
]{scrbook}

%
% Randabstände
%
\usepackage[]{geometry}

%
% Paket Deutsch
%
\usepackage[ngerman]{babel}

%
% Pakete für UTF-8
%
\usepackage[utf8]{inputenc}
\usepackage[T1]{fontenc}

%
% Anführungszeichen
%
\usepackage[babel,german=quotes]{csquotes}

% 
% Verweise mit Autor und Jahr
%
\usepackage{natbib}
\bibliographystyle{abbrvnat}

%
% Paket für Tabelleneigenschaften
%
\usepackage{array}

%
% Paket für Tabellen
%
\usepackage{booktabs}

%
% Paket für Grafiken
%
\usepackage{graphicx}

%
% Spezielle Schrift setzen
%
\setkomafont{sectioning}{\normalfont\bfseries}
\setkomafont{captionlabel}{\normalfont\bfseries} 
\setkomafont{pagehead}{\normalfont\fontsize{9.5pt}{11.4pt}} % Kopfzeilenschrift
\setkomafont{descriptionlabel}{\normalfont\bfseries}

%
% Zeilenumbruch bei Bildbeschreibungen
%
\setcapindent{1em}

%
% Kopf und Fußzeilen
%
\usepackage[automark,autooneside=false]{scrlayer-scrpage}
\pagestyle{scrheadings}
\automark[section]{chapter}
\ihead{\leftmark}
\ohead{\rightmark}
\chead{}

%
% Mathematische Symbole
%
\usepackage{amsmath}
\usepackage{amssymb}

%
% Type 1 Fonts für bessere Darstellung in PDF verwenden.
%
%\usepackage{mathptmx}           % Times
\usepackage{helvet}              % Helvetica
\renewcommand{\familydefault}{\sfdefault}
\usepackage{courier}             % Courier

%
% Paket für Farben im PDF
%
\usepackage{color}

%
% Paket für Links innerhalb des PDF Dokuments
%
\definecolor{LinkColor}{rgb}{0,0,0}
\usepackage[%
	pdftitle={Titel},                        % Titel der Arbeit
	pdfauthor={Autor},                       % AutorIn
	pdfcreator={LaTeX},                      % genutzte Programme
	pdfsubject={Betreff},
	pdfkeywords={Keywords}]{hyperref}
\hypersetup{colorlinks=true, linkcolor=LinkColor, 
            citecolor=LinkColor, filecolor=LinkColor, 
            menucolor=LinkColor, pagecolor=LinkColor, urlcolor=LinkColor}                                       

%
% Paket um Listings zu formatieren
%
\usepackage[savemem]{listings}
\lstloadlanguages{TeX}


%
% Neue Umgebungen
%
\newenvironment{ListChanges}%
	{\begin{list}{$\diamondsuit$}{}}%
	{\end{list}}

%
% aller Bilder im Unterverzeichnis 'bilder' gesucht
%
\graphicspath{{bilder/}}

%
% Strukturiertiefe bis subsubsection{} möglich
%
\setcounter{secnumdepth}{3}

%
% Dargestellte Strukturiertiefe im Inhaltsverzeichnis
%
\setcounter{tocdepth}{3}

%
% Abkürzungsverzeichnis
%
\usepackage[intoc]{nomencl}
\usepackage[toc]{glossaries}
\makeglossaries
\renewcommand*\glspostdescription{\dotfill}
