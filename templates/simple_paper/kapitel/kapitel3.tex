\chapter{Drittes Kapitel}

\section{Aufzählungen und Unter-unterkapitel}

\subsection{einfache Liste}

Diese Methode wird in der Naturwissenschaft als "automatische Übersetzung" bezeichnet, bei der die Übersetzung in eine andere Sprache als in einer anderen Sprache (z.B. in Japanisch, Koreanisch, Deutsch, Englisch, Französisch, Italienisch, Spanisch, Polnisch, Portugiesisch, Russisch, Schwedisch usw.) erfolgen muss, um die gewünschte Zielsprache zu erhalten (siehe oben).

\begin{itemize}
 \item erstes Element
 \item zweites Element
 \item Element Nummer drei
\end{itemize}

Die meisten Maschinen zeichnen sich durch ihre Einfachheit und Einfachheit aus, aber die meisten dieser Maschinen sind nicht für den professionellen Gebrauch bestimmt.

\subsection{nummerierte Liste}

Diese Maschine ist die einzige Maschine, die es erlaubt, jede beliebige Anzahl von Wörtern in einer Sprache zu schreiben, unabhängig davon, ob es sich dabei um eine oder mehrere Wörter handelt, oder ob sie sich um ein oder zwei Wörter oder um zwei oder mehr Wörter handeln muss, um einen Text zu erzeugen.

\begin{enumerate}
 \item erster Punkt
 \item weiterer Punkt
 \item letzter Punkt
\end{enumerate}

Die maschinelle Übersetzung wird durch die Maschine selbst durchgeführt, d.h. durch einen Übersetzer, der das Ergebnis der maschinellen Übersetzung mit Hilfe der Software selbst auswählt, oder durch den Übersetzer in einem anderen Land als dem, in dem er sich aufhält, wo er arbeitet und wo es ihm möglich ist, seine Muttersprache zu sprechen (im Gegensatz zu anderen Sprachen, die in anderen Ländern als Muttersprache gesprochen werden).

\section{Formeln}

Die Maschine besteht aus der Exponentialfunktion 
$e^x = \lim_{n \to \infty} \left( 1+ \frac{x}{n} \right)^n$ 
und dem Codegenerator, der mit dem Computer verbunden 
ist und der Software, mit der der Code generiert werden kann, 
um die Software zu programmieren und zu schreiben, und die 
Maschine kann in verschiedenen Sprachen verwendet werden.

\begin{align}
\label{bayes}
    P(A \mid B) = \frac{P(B \mid A) \, P(A)}{P(B)}
\end{align}

Im Gegensatz zu den meisten anderen Programmen können wir keine Programmiersprache benutzen, weil wir die Sprache nicht beherrschen, sondern nur lernen, sie zu verstehen und sie in andere Sprachen zu übersetzen, damit wir sie verstehen können und mit ihnen arbeiten können (d. h. mit ihr arbeiten)
